\documentclass[draft]{article}
\usepackage[a4paper, top=1.5in, bottom=1.5in, right=1in, left=1in]{geometry}
\usepackage[bottom]{footmisc}

\usepackage{polski}
\usepackage[polish]{babel}

\usepackage{mathtools}
\usepackage{amsfonts}

\title{Rachunek różniczkowy i całkowy}
\date{11.05.2024}
\author{Michał Nycz}

\begin{document}
    \pagenumbering{gobble}
    \maketitle
    \tableofcontents
    \newpage

    \pagenumbering{arabic}
    \section{Pochodne funkcji jednej zmiennej}

    \subsection{Wzory na pochodne podstawowych funkcji}

    \paragraph{Pochodna funckji stałej:}
    \[ (c)^\prime = 0 \text{, \quad gdzie} \; c \in \mathbb{R} \text{ jest stałą} \]
    
    \paragraph{Pochodna funckji potęgowej:}
    \[ (x^a)^\prime = ax^{a-1} \text{, \quad gdzie} \; a \in \mathbb{R} \text{ jest stałą} \]
    
    \paragraph{Pochodna funckji wykładniczej i logarytmicznej:}
    \begin{align*}
        (a^x)^\prime = a^x \ln a &\text{, \quad gdzie} \; a \in (0, 1) \cup (1, \infty) \text{ jest stałą} \\
        (\log_a x)^\prime = \frac{1}{x \ln a} &\text{, \quad gdzie} \; a \in (0, 1) \cup (1, \infty) \text{ jest stałą} \\
        \\
        (e^x)^\prime = e^x \\
        (\ln x)^\prime = \frac{1}{x}
    \end{align*}

    \paragraph{Pochodna funckji trygonometrycznych:}
    \begin{align*}
        (\sin x)^\prime &= \cos x \\
        (\cos x)^\prime &= - \sin x \\
        (\text{tg } x)^\prime &= \frac{1}{\cos^2 x} \\
        (\text{ctg } x)^\prime &= - \frac{1}{\sin^2 x} \\
    \end{align*}

    \paragraph{Pochodne funkcji łączonych:}
    \begin{align*}
        (a \cdot f)^\prime &= a \cdot f^\prime \text{, \quad gdzie} \; c \in \mathbb{R} \text{ jest stałą} \\
        \\
        (f \pm g)^\prime &= f^\prime \pm g^\prime \\
        (f \cdot g)^\prime &= f^\prime g + f g^\prime \\
        (\frac{f}{g})^\prime &= \frac{f^\prime g - f g^\prime}{g^2}
    \end{align*}

    \paragraph[Przydatne pochodne:]
        {Przydatne pochodne\footnote{Przydatne pochodne wywodządze się z pochodnych funkcji podstawowych}:}
    \begin{align*}
        (x)^\prime &= 1 \\
        (ax)^\prime &= a \\
        (\sqrt{x})^\prime &= \frac{1}{2\sqrt{x}} \\
    \end{align*}

\end{document}