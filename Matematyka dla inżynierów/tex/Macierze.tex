\documentclass[../Matematyka.tex]{subfiles}

\begin{document}
    \section{Macierze}
    Macierz tworzą liczby wpisane do prostokątnej tabelki

    \begin{displaymath}
        A_{m,n} = 
        \begin{bmatrix}
            a_{1,1} & a_{1,2} & \cdots & a_{1,n} \\
            a_{2,1} & a_{2,2} & \cdots & a_{2,n} \\
            \vdots & \vdots & \ddots & \vdots \\
            a_{m,1} & a_{m,2} & \cdots & a_{m,n} \\
        \end{bmatrix}
    \end{displaymath}

    \begin{displaymath}
        a_{ij} \qquad
        \substack{
            i \text{- numer wiersza}\\
            j \text{- numer kolumny}
        }
    \end{displaymath}

    \(M^{m \times n}\) - Zbiór wszystkich macierzy wymiaru \(m \times n\).

    \subsection{Szczególne typy macierzy}

    \subsubsection{Macierz zerowa}
    Macierz złożona z samych zer.

    \begin{displaymath}
        \theta_{3 \times 2} =
        \begin{bmatrix}
            0 & 0 \\
            0 & 0 \\
            0 & 0 \\
        \end{bmatrix}
    \end{displaymath}

    \subsubsection{Macierz kwadratowa}
    Macierz w której liczba wierszy równa się liczbie kolumn (\(m = n\))

    Wyróżniamy główną przekątną:
    \begin{displaymath}
        \begin{bmatrix}
            \ddots & \\
            & \ddots
        \end{bmatrix}
    \end{displaymath}
    \[(a_{1,1}, \; a_{2,2}, \;\dots\; a_{n,n})\]

    \subsubsection{Macierz trójkątna}
    To macierz kwadratowa w której wszystkie elementy nad lub pod główną przekątną wynoszą zero.

    \begin{displaymath}
        \begin{bmatrix}
            \ddots &  0 \\
            \cdots & \ddots
        \end{bmatrix}
        \text{ - Macierz trójkątna dolna.}
    \end{displaymath}

    \begin{displaymath}
        \begin{bmatrix}
            \ddots & \cdots \\
            0 & \ddots
        \end{bmatrix}
        \text{ - Macierz trójkątna górna.}
    \end{displaymath}

    \subsubsection{Macierz diagonalna}
    To macierz, która jest trójkątna górna i dolna.
    Inaczej mówiąc jest to macierz kwadratowa w której poza główną przekątną występują same zera.

    \begin{displaymath}
        \begin{bmatrix}
            \ddots & 0 \\
            0 & \ddots
        \end{bmatrix}
    \end{displaymath}

    \subsubsection{Macierz jednostkowa}
    To macierz diagonalna w której na głównej przekątnej występują same \(1\).

    \begin{displaymath}
        I =
        \begin{bmatrix}
            1 & &         & 0 \\
            & 1 &         &   \\
            &   & \ddots  &   \\
            0 & &         & 1 \\
        \end{bmatrix}
    \end{displaymath}
    \begin{displaymath}
        I_1 =
        \begin{bmatrix}
            1
        \end{bmatrix}
    \end{displaymath}
    \begin{displaymath}
        I_3 =
        \begin{bmatrix}
            1 & 0 & 0 \\
            0 & 1 & 0 \\
            0 & 0 & 1 \\
        \end{bmatrix}
    \end{displaymath}

    \newpage
    \subsection{Działania na macierzach}

    \subsubsection{Transponowanie (Transpozycja)}
    \begin{displaymath}
        A \in M^{m\;\times\;n},\quad
        B \in M^{n\;\times\;m} \qquad
        \substack
        {
            i \in \{1,2, \cdots,m\}\\
            j \in \{1,2, \cdots,n\}
        }
    \end{displaymath}
    \[B = A^T \iff b_{ji} = a_{ij}\]

    Aby transponować macierz \(A\) należy zamienić wiersze macierzy \(A\) na kolumny (albo kolumny na wiersze).

    \begin{displaymath}
        A = 
        \begin{bmatrix}
            3 & 0 & 5 \\
            4 & 7 & 1
        \end{bmatrix}
        \qquad
        A^T =
        \begin{bmatrix}
            3 & 4 \\
            0 & 7 \\
            5 & 1
        \end{bmatrix}
    \end{displaymath}

    \subsubsection{Mnożenie macierzy przez liczbę}
    \begin{displaymath}
        A,B \in M^{m\;\times\;n},\quad
        \alpha \in R \qquad
        \substack
        {
            i \in \{1,2, \cdots,m\}\\
            j \in \{1,2, \cdots,n\}
        }
    \end{displaymath}
    \[B = \alpha \cdot A \iff b_{ij} = \alpha \cdot a_{ij}\]

    Aby pomnożyć Macierz \(A\) przez liczbę \(\alpha\) każdy element macierzy \(A\) mnożymy przez liczbę \(\alpha\).

    \begin{displaymath}
        A = 
        \begin{bmatrix}
            3  & 5  \\
            0  & -1 \\
            -4 & 8
        \end{bmatrix}\qquad
        3A = 
        \begin{bmatrix}
            9   & 15 \\
            0   & -3 \\
            -12 & 24
        \end{bmatrix}
    \end{displaymath}

    \subsubsection{Dodawanie i odejmowanie macierzy}
    \begin{displaymath}
        A,B,C,D \in M^{m\;\times\;n}\qquad
        \substack
        {
            i \in \{1,2, \cdots,m\}\\
            j \in \{1,2, \cdots,n\}
        }
    \end{displaymath}
    \begin{align*}
        C &= A + B \iff c_{ij} = a_{ij} + b_{ij}\\
        D &= A - B \iff c_{ij} = a_{ij} - b_{ij}
    \end{align*}

    Dodawanie i odejmowanie można wykonać tylko na macierzach tego samego wymiaru.\par
    Działania te wykonujemy na współrzędnych to znaczy dodajemy/odejmujemy liczby na tych samych pozycjach.

    \begin{displaymath}
        A =
        \begin{bmatrix}
             4 & 0 & -3 \\
            -2 & 5 & 1 \\
        \end{bmatrix}\quad
        B =
        \begin{bmatrix}
            -7 & 6 & 4 \\
            -9 & 8 & 0 \\
        \end{bmatrix}
    \end{displaymath}

    \begin{displaymath}
        A + B =
        \begin{bmatrix}
             4 & 0 & -3 \\
            -2 & 5 & 1 \\
        \end{bmatrix}
        +
        \begin{bmatrix}
            -7 & 6 & 4 \\
            -9 & 8 & 0 \\
        \end{bmatrix}
        =
        \begin{bmatrix}
            -3  & 6  & 1 \\
            -11 & 13 & 1 \\
        \end{bmatrix}
    \end{displaymath}

    \begin{displaymath}
        A - B =
        \begin{bmatrix}
             4 & 0 & -3 \\
            -2 & 5 &  1 \\
        \end{bmatrix}
        -
        \begin{bmatrix}
            -7 & 6 & 4 \\
            -9 & 8 & 0 \\
        \end{bmatrix}
        =
        \begin{bmatrix}
            11 & -6 & -7 \\
            7  & -3 &  1 \\
        \end{bmatrix}
    \end{displaymath}

    \begin{displaymath}
        B - A =
        \begin{bmatrix}
            -11 & 6 &  7 \\
            -7  & 3 & -1 \\
        \end{bmatrix}
    \end{displaymath}

    \subsubsection{Mnożenie macierzy}
    \begin{displaymath}
        A \in M^{m \times p},\quad
        B \in M^{p \times n},\quad
        C \in M^{m \times n}
        \qquad
        \substack
        {
            i \in \{1,2, \cdots,m\}\\
            j \in \{1,2, \cdots,n\}
        }
    \end{displaymath}
    \[C = A \cdot B \iff c_{ij} = \sum_{k=1}^p a_{ik} \cdot b_{kj}\]

    Aby wykonać mnożenie \(A\) razy \(B\) liczba kolumn macierzy \(A\) musi być równa liczbie wierszy macierzy \(B\).

    \begin{displaymath}
        \begin{bmatrix}
            a_1, a_2, \cdots\!, a_n 
        \end{bmatrix}
        \cdot
        \begin{bmatrix}
            b_1\\ 
            b_2\\
            \vdots\\
             b_n 
        \end{bmatrix} =
        a_1b_1 + a_2b_2 + \dots + a_nb_n
    \end{displaymath}

    Aby wykonać mnożenie \(A \cdot B\) pierwszy wiersz \(A\) mnożymy przez wszystkie kolumny \(B\), następnie drugi wiersz \(A\) przez wszystkie kolumny \(B\) i tak dalej.

    \begin{displaymath}
        A =
        \begin{bmatrix}
            4 & 0 & -2 \\
            1 & 5 & -1
        \end{bmatrix}\quad
        B =
        \begin{bmatrix}
            3 & 1 \\
            0 & 2 \\
            4 & 0
        \end{bmatrix}
    \end{displaymath}

    \begin{displaymath}
        A \cdot B = 
        \begin{bmatrix}
            4 & 0 & -2 \\
            1 & 5 & -1
        \end{bmatrix}\cdot
        \begin{bmatrix}
            3 & 1 \\
            0 & 2 \\
            4 & 0
        \end{bmatrix} =
        \begin{bmatrix}
            4\cdot3+0\cdot0+4\cdot(-2) & 4\cdot1+0\cdot2+(-2)\cdot0 \\
            1\cdot3+5\cdot0+(-1)\cdot4 & 1\cdot1+5\cdot2+(-1)\cdot0 
        \end{bmatrix} = 
        \begin{bmatrix}
            4 & 4 \\
            -1 & 11 
        \end{bmatrix} 
    \end{displaymath}

    \begin{displaymath}
        B \cdot A = 
        \begin{bmatrix}
            3 & 1 \\
            0 & 2 \\
            4 & 0
        \end{bmatrix}\cdot
        \begin{bmatrix}
            4 & 0 & -2 \\
            1 & 5 & -1
        \end{bmatrix} =
        \begin{bmatrix}
            3\cdot4+1\cdot1 & 3\cdot0+1\cdot5 & 3\cdot(-2)+3\cdot(-1) \\
            0\cdot4+2\cdot1 & 0\cdot0+2\cdot5 & 0\cdot(-2)+2\cdot(-1) \\
            4\cdot4+0\cdot1 & 4\cdot0+0\cdot5 & 4\cdot(-2)+0\cdot(-1) 
        \end{bmatrix} = 
        \begin{bmatrix}
            13 & 5 & -9 \\
            2 & 10 & -2 \\
            16 & 0 & -8 
        \end{bmatrix} 
    \end{displaymath}

    \subsection{Wyznacznik macierzy}
    Wyznacznik to liczba przyporządkowana macierzy kwadratowej.

    Macierze kwadratowe dzielimy na:
    \begin{itemize}
        \item osobliwe, tzn. \(detA = 0\).
        \item nieosobliwe, tzn. \(detA \neq 0\)
    \end{itemize}

    \[detA \text{ - wyznacznik}\]

    \begin{displaymath}
        n=1
        \quad
        A =
        \begin{bmatrix}
            a
        \end{bmatrix}
        \quad
        detA = a
    \end{displaymath}

    \begin{displaymath}
        n=2
        \quad
        A =
        \begin{bmatrix}
            a & b \\
            c & d
        \end{bmatrix}
        \quad
        detA = ad - bc
    \end{displaymath}

    \subsubsection{Wzory Sarrusa}
    Używane do liczenia wyznacznika dla macierzy kwadratowej rozmiaru \(n = 3\).

    \begin{displaymath}
        det
        \substack{
            \begin{bmatrix}
                1 & 0 & 3 \\
                0 & 2 & 5 \\
                4 & 1 & 6
            \end{bmatrix}\\
            \begin{matrix}
                1 & 0 & 3 \\
                0 & 2 & 5
            \end{matrix}
        } =
        (1 \cdot 2 \cdot 6 + 0 \cdot 1 \cdot 3 + 4 \cdot 0 \cdot 5) - (4 \cdot 2 \cdot 3 + 1 \cdot 1 \cdot 5 + 0 \cdot 0 \cdot 6) = 12 - 29 = -17
    \end{displaymath}

    \subsubsection{Tw. Laplace'a}
    Jeżeli \(A\) jest macierzą kwadratową wymiaru \(n \geq 2\), to 

    \begin{displaymath}
        detA =
        \sum^{n}_{j = 1}
        a_{ij}D_{ij}
        \qquad
        \text{Rozwinięcie względem wiersza} \; i
    \end{displaymath}
    \begin{displaymath}
        detA =
        \sum^{n}_{i = 1}
        a_{ij}D_{ij}
        \qquad
        \text{Rozwinięcie względem kolumny} \; j
    \end{displaymath}

    Gdzie \(D_{ij}\) to dopełnienie algebraiczne \(a_{ij}\)
    \[D_{ij} = (-1)^{i+j} \cdot detA_{ij}\]
    gdzie \(A_{ij}\) to macierz, która powstaje z \(A\) przez skreślenie wiersza \(i\) oraz kolumny \(j\);
    
    Stosując wzór Laplace'a szukamy wiersza lub kolumny z największą ilością zer.
    Jeżeli w maceirzy występuje wiersz lub kolumna złożona z samych zer to \(detA = 0\).

    \subsection{Macierz odwrotna}
    Macierz \(A^-1\) jest macierzą odwrotną do \(A\), jeżeli:
    \[A^{-1} \cdot A = A \cdot A^{-1} = I\]

    Macierz \(A\) jest odwracalna \(\iff A\) jest maceirzą nieosobliwą.

    Jeżeli \(A\) jest macierzą kwadratową nieosobliwą wymiaru wymiaru \(n \geq 2\), to
    \[A^{-1} = \frac{1}{detA} \cdot D^T\]
    gdzie \(D = \begin{bmatrix} D_{ij} \end{bmatrix}\) jest macierzą dopełnień algebraicznych \(a\)

    Uwaga: \(n = 1\)
    \[A = \begin{bmatrix} 5 \end{bmatrix} \qquad A^{-1} = \begin{bmatrix} \frac{1}{5} \end{bmatrix} \]

    \newpage
    \subsection{Minor macierzy}
    Minor \(M\) macierzy \(A\) to macierz kwadratowa, która pwostaje z \(A\) przez skreślenie pewnej ilości (być może zero) wierszy i kolumn.

    \subsubsection{Minor bazowy}
    Minor \(M\) macierzy \(A\) jest minorem bazowym \(A\), jeżeli \(detM \neq 0\) oraz wszystkie minory \(M'\) macierzy \(A\) wymiaru większego niż \(M\) mają wyznaczniki równe zero (\(detM' = 0\))

    Uwaga. Macierz \(A\) może posiadać więcej niż jeden minor bazowy, ale wszystkie minory bazowe \(A\) są tego samego wymiaru.

    \subsubsection{Rząd macierzy}
    Rząd macierzy niezerowej to wymiar dowlonego minora bazowego tej macierzy.

    Rząd macierzy zerowej wynosi zero (\(rz(\theta) = 0\))

    \subsection{Układy równań liniowych}
    Z układem równań liniowych można powiązać macierz \(A\) wymiaru \(m \times n\) nazywaną \textbf{macierzą współczynników układu równań} (macierz główna) 
    oraz dwie macierze kolumnowe \(x\) (kolumna zmiennych) i \(b\) (kolumna wyrazów wolnych).
    \[Ax = b\]

    \subsubsection{Układ Cramera}
    Układ równań liniowych jest układem Cramera, jeżeli macierz główna układu \(A\) jest kwadratowa nieosobliwa.
    Czyli liczba równań w układzie jest równa liczbie zmiennych a wyznacznik macierzy głównej nie jest równy 0 (\(detA \neq 0\)).

    \subsubsection{Metoda macierzy odwrotnej}
    Układ równań liniowych Cramera ma dokładnie jedno rozwiązanie zadane wzorem
    \[x = A^{-1} \cdot b\]

    \subsubsection{Metoda Cramera}
    Układ równań liniowych Cramera ma dokładnie jedno rozwiązanie zadane wzorem
    \begin{displaymath}
        x_i = \frac{detA_i}{detA},
        \qquad
        i = 1, 2, \dots, n
    \end{displaymath}
    gdzie \(A_i\) to macierz, która powstaje z \(A\) przez zastąpienie kolumny \(i\) przez kolumnę wyrazów wolnych.

    \subsubsection{Twierdzenie Kroneckera-Capelliego}
    Macierz $`U`$ (macierz uzupełniona) powstaje z macierzy $`A`$ przez dołączenie kolumny $`b`$.
    \begin{displaymath}
        U =
        \begin{bmatrix}
            A & \vdots & b
        \end{bmatrix}
    \end{displaymath}

    \begin{itemize}
        \item jeżeli \(rz(A) = rz(U) = n\) (\(n\) - liczba niewiadowym), to rozwiązanie jest jedyne;
        \item jeżeli \(rz(A) = rz(U) = r < n\), to rozwiązań jest nieskończenie wiele i zależą od \(n - r\) parametrów.
    \end{itemize}

    \subsubsection{Macierz schodkowa}
    Macierz schodkowa to macierz w której każdy pierwszy nie zerowy element wiersza jest przesunięty w prawo w stosunku do wiersza poprzedniego

    Nie bierzemy pod uwagę wierszy zerowych

    \begin{displaymath}
        A =
        \begin{bmatrix}
            5 & 0 & 0 & 0 \\
            0 & 4 & 0 & 0 \\
            0 & 0 & 0 & 3
        \end{bmatrix}
        \quad
        rz(A) = 3
    \end{displaymath}

    Rząd macierzy schodkowej jest równy liczbie schodków to znaczy nie zerowych wierszy.

    \subsubsection{Operacje elementarne na wierszach}
    \begin{enumerate}
        \item Pomnożyć wiersz przez liczbę różną od zera,
        \item do wiersza dodać inny wiersz pomnożony przez liczbę,
        \item zamienić dwa wiersze miejscami.
    \end{enumerate}

    Analogiczne operacje definiujemy dla kolumn

    Przekształcenia elementarne nie zmieniają rzędu macierzy.

    \subsubsection{Metoda Gaussa (eliminacji zmiennych)}
    Przekształcamy macierz uzupełniną układu za pomocą operacji elementarnych na wierszach do postaci schodkowej. 
    Z przekształconej macierzy odczytujemy czy rząd \(A\) jest równy rzędowy \(U\), jeżeli nie to układ jest sprzeczny, 
    jeżeli są równe to z przekształconej macierzy odczytujemy równania układu a następnie rozwiązania. 
\end{document}