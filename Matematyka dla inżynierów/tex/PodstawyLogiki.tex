\documentclass[../Matematyka.tex]{subfiles}

\begin{document}
    \section{Podstawy logiki matematycznej}

    \subsection{Elementy logiki matematycznej}

    \subsubsection*{Negatyw \(\sim\)}
    \begin{tabular}{c|>{\columncolor[gray]{.8}}c}
        \(p\) & \(\sim \! p\) \\
        \hline
        \(1\) & \(0\) \\
        \(0\) & \(1\) \\
    \end{tabular}

    \subsubsection*{Alternatywa \(\lor\)}
    \begin{tabular}{c|c|>{\columncolor[gray]{.8}}c}
        \(p\) & \(p\) & \(p \lor q\) \\
        \hline
        \(1\) & \(1\) & 1 \\
        \(1\) & \(0\) & 1 \\
        \(0\) & \(1\) & 1 \\
        \(0\) & \(0\) & 0 \\
    \end{tabular}

    \subsubsection*{Koniunkcja \(\land\)}
    \begin{tabular}{c|c|>{\columncolor[gray]{.8}}c}
        \(p\) & \(p\) & \(p \land q\) \\
        \hline
        \(1\) & \(1\) & 1 \\
        \(1\) & \(0\) & 0 \\
        \(0\) & \(1\) & 0 \\
        \(0\) & \(0\) & 0 \\
    \end{tabular}

    \subsubsection*{Implikacja \(\implies\)}
    \begin{tabular}{c|c|>{\columncolor[gray]{.8}}c}
        \(p\) & \(p\) & \(p \implies q\) \\
        \hline
        \(1\) & \(1\) & 1 \\
        \(1\) & \(0\) & 0 \\
        \(0\) & \(1\) & 1 \\
        \(0\) & \(0\) & 1 \\
    \end{tabular}

    \subsubsection*{Równoważność \(\iff\)}
    \begin{tabular}{c|c|>{\columncolor[gray]{.8}}c}
        \(p\) & \(p\) & \(p \iff q\) \\
        \hline
        \(1\) & \(1\) & 1 \\
        \(1\) & \(0\) & 0 \\
        \(0\) & \(1\) & 0 \\
        \(0\) & \(0\) & 1 \\
    \end{tabular}

    \newpage
    \subsection{Prawa logiczne}

    \textbf{Tautologia} - Zdanie zawsze prawdziwe.

    \subsubsection{Prawa De Morgana}
    \subsubsection*{I prawo De Morgana}
    Prawo zaprzeczania koniunkcji: negacja koniunkcji jest równoważna alternatywie negacji
    \[[\sim\!(p \land q)] \iff (\sim\!p \; \lor \sim\!q)\]

    \begin{table}[H]
        \centering
        \caption{Wartości logiczne I prawa De Morgana}
        \begin{tabular}{c|c|>{\columncolor[gray]{.8}}c|c|c|c|>{\columncolor[gray]{.8}}c}
            \(p\) & \(q\) & \(p \land q\) & \(\sim\!(p \land q)\) & \(\sim\!p\) & \(\sim\!q\) & \((\sim\!p) \lor (\sim\!q)\) \\
            \hline
            1 & 1 & 1 & 0 & 0 & 0 & 0 \\
            1 & 0 & 0 & 1 & 0 & 1 & 1 \\
            0 & 1 & 0 & 1 & 1 & 0 & 1 \\
            0 & 0 & 0 & 1 & 1 & 1 & 1 \\
        \end{tabular}
    \end{table}

    \subsubsection*{II prawo De Morgana}
    Prawo zaprzeczenia alternatywy: negacja alternatywy jest równoważna koniunkcji negacji
    \[[\sim\!(p \lor q)] \iff (\sim\!p \; \land \sim\!q)\]

    \begin{table}[H]
        \centering
        \caption{Wartości logiczne II prawa De Morgana}
        \begin{tabular}{c|c|>{\columncolor[gray]{.8}}c|c|c|c|>{\columncolor[gray]{.8}}c}
            \(p\) & \(q\) & \(p \lor q\) & \(\sim\!(p \lor q)\) & \(\sim\!p\) & \(\sim\!q\) & \((\sim\!p) \land (\sim\!q)\) \\
            \hline
            1 & 1 & 1 & 0 & 0 & 0 & 0 \\
            1 & 0 & 1 & 0 & 0 & 1 & 0 \\
            0 & 1 & 1 & 0 & 1 & 0 & 0 \\
            0 & 0 & 0 & 1 & 1 & 1 & 1 \\
        \end{tabular}
    \end{table}

    \subsubsection{Prawo kontrapozycji}
    \[(p \implies q) \iff (\sim\!p \implies \sim\!q)\]

    \begin{table}[H]
        \centering
        \caption{Wartości logiczne prawa kontrapozycji}
        \begin{tabular}{c|c|>{\columncolor[gray]{.8}}c|c|c|>{\columncolor[gray]{.8}}c}
            \(p\) & \(q\) & \(p \implies q\) & \(\sim\!q\) & \(\sim\!p\) & \((\sim\!q) \implies (\sim\!p)\) \\
            \hline
            1 & 1 & 1 & 0 & 0 & 1 \\
            1 & 0 & 0 & 1 & 0 & 0 \\
            0 & 1 & 1 & 0 & 1 & 1 \\
            0 & 0 & 1 & 1 & 1 & 1 \\
        \end{tabular}
    \end{table}
\end{document}