\documentclass[../Matematyka.tex]{subfiles}

\begin{document}
\section{Rachunek prawdopodobieństwa i kombinatoryka}

\subsection{Elementy kombinatoryki}
    \subsubsection{Symbol sumy}
    \(\sum\) - sigma, symbol sumy
    \[\sum^{5}_{i=2} i^2 = 2^2 + 3^2 + 4^2 + 5^2 = 4 + 9 + 16 + 25 = 54\]

    \subsubsection{Symbol iloczynu}
    \(\prod\) - pi, symbol iloczynu
    \[\prod^{n}_{i=1} i = 1 \cdot 2 \cdot 3 \cdot \ldots \cdot n = n!\]

    \subsubsection{Silnia}
    \(n!\) - \(n\) silnia \(\qquad n \in \mathbb{N}_0\)

    \begin{displaymath}
        n! =
        \begin{cases}
            1, & \quad n = 0 \lor n = 1 \\
            1 \cdot 2 \cdot \ldots \cdot n, & \quad n > 1
        \end{cases}
    \end{displaymath}

    \[n! = (n - 1)! \cdot n, \quad n \in \mathbb{N}\]

    Def. Permutacja skończonego zbirou \(A\) to ciąg wszystkich elementów zbioru \(A\).\par
    Tw. Ilość wszystkich permutacji zbioru \(n\)-elementowego wynosi \(n!\).

    \subsubsection{Symbol Newtona}

    \begin{align*}
        \binom{n}{k} = 
        \frac{n!}{k!(n-k)!} \qquad&
        \substack{
            k,\;n\;\in\;\mathbb{N}_0 \\
            k\;\leq\;n
        }\\
        \binom{n}{n-k} = 
        \binom{n}{k} \qquad&
        \substack{
            k,\;n\;\in\;\mathbb{N}_0 \\
            k\;\leq\;n
        }
    \end{align*}

    \begin{displaymath}
        \binom{n}{0} = 1 \quad
        \binom{n}{1} = n \quad
        \binom{n}{n - 1} = n \quad
        \binom{n}{n} = 1
    \end{displaymath}

    Tw. Ilość wszystkich \(k\)-elementowych podzbiorów zbioru \(n\)-elementowego wynosi \(\binom{n}{k}\)\par
    Tw. Ilość wszystkich podzbiorów zbioru \(n\)-elementowego \(2^n\)\par
    Tw. Dla \(k, n \in \mathbb{N}, k \leq n\)
    \begin{displaymath}
        \binom{n}{k-1}+
        \binom{n}{k} = 
        \binom{n+1}{k}
    \end{displaymath}

    \subsubsection*{Dwumian Newtona}
    Tw. Dwumian Newtona, dla  \(a, b \in \mathbb{R} \;\text{i}\; n \in \mathbb{N}\)

    \[(a+b)^n = \sum^{n}_{k=0} \binom{n}{k}a^{n-k}b^k\]

    \subsubsection*{Trójkąt Pascala}

    \[\binom{0}{0}\]
    \[\binom{1}{0}\quad\binom{1}{1}\]
    \[\binom{2}{0}\quad\binom{2}{1}\quad\binom{2}{2}\]
    \[\binom{3}{0}\quad\binom{3}{1}\quad\binom{3}{2}\quad\binom{3}{3}\]
    \[\binom{4}{0}\quad\binom{4}{1}\quad\binom{4}{2}\quad\binom{4}{3}\quad\binom{4}{4}\]
    \[\binom{5}{0}\quad\binom{5}{1}\quad\binom{5}{2}\quad\binom{5}{3}\quad\binom{5}{4}\quad\binom{5}{5}\]
    \[\binom{6}{0}\quad\binom{6}{1}\quad\binom{6}{2}\quad\binom{6}{3}\quad\binom{6}{4}\quad\binom{6}{5}\quad\binom{6}{6}\]

    \[1\]
    \[1\quad1\]
    \[1\quad2\quad1\]
    \[1\quad3\quad3\quad1\]
    \[1\quad4\quad6\quad4\quad1\]
    \[1\quad5\quad10\quad10\quad5\quad1\]
    \[1\quad6\quad15\quad20\quad15\quad6\quad1\]

    \subsubsection{Regóła mnożenia}
    Jeżeli pewien wybór zależy od skończenie wielu decyzji, powiedzmy \(k\), 
    przy czym podejmując pierwszą decyzję mamy \(n_1\) możliwości, drugą \(n_2\) możliwości, 
    \(\ldots\), \(k\)-tą \(n_k\) możliwości, bo wybór ten może być zrobiony na:

    \[n = n_1 \cdot n_2 \cdot \ldots \cdot n_k\]

    \subsubsection{Permutacja bez powtórzeń}
    \textit{Permutacja bez powtórzeń} zbioru \(n\)-elementowego \(A = \{a_1, a_2, \ldots, a_n\}\), dla
    \(n \in \N\) nazywamy każdy \(n\)-wyrazowy ciąg utworzony ze wszystkich \(n\)-elementów zbioru
    \(A\), czyli każde uporządkowanie elementów zbioru \(A\).

    Liczba wszystkich różnych permutacji bez powtórzeń zbioru \(n\)-elementowego jest równa
    \[P_n = n!\]

    Permutacje wykorzystujemy, gdy:
    \begin{itemize}
        \item występują wszystkie elementy zbioru,
        \item kolejność jest istotna.
    \end{itemize}

    \subsubsection{Permutacja z powtórzeniami}
    \textit{Permutacją \(n\)-wyrazową z powtórzeniami} zbioru \(k\)-elementowego
    \(A = \{a_1, a_2, \ldots, a_k\}\), w której element \(a_i\) występuje \(n_i\) razy, \(i = 1, 2, \ldots, k\), przy czym \(\sum_{i=0}^{k}n_i = n\).

    Liczba wszystkich różnych \(n\)-wyrazowych permutacji z powtórzeniami ze zbioru \(k\)-elementowego jest równa:
    \[P_n(n_1, n_2, \ldots, n_k) = \frac{n!}{n_1! \cdot n_2! \cdot \ldots \cdots n_k!},\]
    \[\text{gdzie } n_i \in \N, i=1,2,\ldots,k \text{, } n_i \text{ - liczba powtórzeń elementu } a_i \in A, \sum_{i=0}^{k}n_i = n\]

    \subsubsection{Wariacja z powtórzeniami}
    \textit{Wariacją \(k\)-wyrazową z powtórzeniami} zbioru \(A\), \(n\)-elementowego, gdzie \(k\in\N\), 
    nazywamy każdy \(k\)-wyrazowy ciąg, którego wyrazami są elementy danego zbioru \(A\).

    Liczba wszystkich różnych \(k\)-wyrazowych wariacji z powtórzeniami zbioru \(n\)-elementowego jest równa:
    \[W_n^k=n^k\]

    Wariacje z powtórzeniami wykorzystujemy, gdy:
    \begin{itemize}
        \item kolejność elementów jest istotna,
        \item elementy mogą się powtarzać (losowanie ze zwracaniem),
        \item niekoniecznie wszystkie elementy zbioru są wykorzystane.
    \end{itemize}

    \newpage
    \subsubsection{Wariacja bez powtórzeń}
    \textit{Wariacją \(k\)-wyrazową bez powtórzeń} zbioru \(A\), \(n\)-elementowego, gdzie \(k\in\N\), 
    nazywamy każdy \(k\)-wyrazowy ciąg różnowartościowy, którego wyrazami są elementy danego zbioru \(A\).

    Liczba wszystkich różnych \(k\)-wyrazowych wariacji bez powtórzeń zbioru \(n\)-elementowego jest równa
    \[V_n^k=\frac{n!}{(n-k)!}\]

    Wariacje bez powtórzeń wykorzystujemy, gdy:
    \begin{itemize}
        \item kolejność elementów jest istotna,
        \item elementy nie mogą się powtarzać (losowanie bez zwracania),
        \item niekoniecznie wszystkie elementy zbioru są wykorzystane.
    \end{itemize}

    \subsubsection{Kombinacja bez powtórzeń}
    \textit{Kombinacją \(k\)-elemntową bez powtórzeń} zbioru \(A\), \(n\)-elementowego, gdzie \(k\in\N\), 
    nazywamy każdy podzbiór \(k\)-elementowy zbioru \(A\), przy czym elementy nie mogą się powtarzać.

    Liczba wszystkich różnych kombinacji \(k\)-elementowych bez powtórzeń jest równa:
    \[C_n^k=\binom{n}{k}=\frac{n!}{k!(n-k)!}\]
    \[\mathcal{A}\]

    \subsection{Prawdopodobieństwo klasyczne}
    \begin{align*}
        \omega& \text{ - zdarzenie elementarne,}\\
        \Omega& \text{ - zbiór wszystkich zdarzeń elementarnych,}\\
        A& \text{ - zdarzenie losowe, } A \subset \Omega,\\
        \mathcal{A} & \text{ - zbiór wszystkich zdarzeń losowych,}\\
        \emptyset& \text{ - zdarzenie niemożliwe,}\\
        \Omega& \text{ - zdarzenie pewne,}\\
        A\prim& \text{ - zdarzenie przeciwne, } A\prim = \Omega \setminus A,\\
    \end{align*}

    Rodzinę podzbiorów \(\mathcal{A}\) zbioru \(\Omega\) nazywamy algebrą zbiorów, jeżeli:
    \begin{enumerate}[label=(\roman*)]
        \item \(A, B \in \mathcal{A} \implies A \cup B \in \mathcal{A}\),
        \item \(A, B \in \mathcal{A} \implies A \cap B \in \mathcal{A}\),
        \item \(A \in \mathcal{A} \implies A\prim = (\Omega \setminus A) \in \mathcal{A}\),
        \item \(\Omega \in \mathcal{A}, \emptyset \in \mathcal{A}\).
    \end{enumerate}

    \newpage
    Prawdopodobieństwo \(P(A)\) to liczba przypisana zdarzeniu losowemu
    \begin{enumerate}[label=(\roman*)]
        \item \(A \cap B = \emptyset \implies P(A \cup B) = P(A) + P(B)\),
        \item \(A \subset B \implies P(B \setminus A) = P(B) - P(A)\),
        \item \(P(A\prim) = 1 - P(A)\),
        \item \(P(\emptyset) = 0\), \(P(\Omega) = 1\),
        \item \(A \subset B \implies P(A) \leq P(B)\),
        \item \(P(A) \in [0,1]\),
        \item \(P(A \cup B) = P(A) + P(B) - P(A \cap B)\).
    \end{enumerate}

    W modelu klasycznym prawdopodobieństwa zakładamy, że zbiór \(\Omega\) jest skończony i wszystkie zdarzenia elementarne są jednakowo prawdopodobne.
    
    Zdarzenia losowe to wszystkie podzbiory zbioru \(\Omega\) i prawdopodobieństwo określa się wzorem:
    \[P(A) = \frac{|A|}{|\Omega|}, \qquad |x| \text{ - ilość elementów }x\] 
\end{document}