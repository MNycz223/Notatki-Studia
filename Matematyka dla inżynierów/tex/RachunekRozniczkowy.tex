\documentclass[../Matematyka.tex]{subfiles}

\begin{document}
    \newcommand{\bis}{^{\prime\prime}}
    \newcommand{\prim}{^\prime}

    \section{Rachunek różniczkowy}

    \subsection{Pochodne funkcji jednej zmiennej}

    \subsubsection{Wzory na pochodne podstawowych funkcji}

    \paragraph{Pochodna funckji stałej:}
    \[ (c)\prim = 0 \text{, \quad gdzie} \; c \in \mathbb{R} \text{ jest stałą} \]
    
    \paragraph{Pochodna funckji potęgowej:}
    \[ (x^\alpha)\prim = \alpha x^{\alpha-1} \text{, \quad gdzie} \; \alpha \in \mathbb{R} \text{ jest stałą} \]
    
    \paragraph{Pochodna funckji wykładniczej i logarytmicznej:}
    \begin{align*}
        (a^x)\prim = a^x \ln a &\text{, \quad gdzie} \; a \in (0, 1) \cup (1, \infty) \text{ jest stałą} \\
        (\log_\alpha x)\prim = \frac{1}{x \ln \alpha} &\text{, \quad gdzie} \; \alpha \in (0, 1) \cup (1, \infty) \text{ jest stałą} \\
        \\
        (e^x)\prim = e^x \\
        (\ln x)\prim = \frac{1}{x}
    \end{align*}

    \paragraph{Pochodna funckji trygonometrycznych:}
    \begin{align*}
        (\sin x)\prim &= \cos x \\
        (\cos x)\prim &= - \sin x \\
        (\tan x)\prim &= \frac{1}{\cos^2 x} \\
        (\cot x)\prim &= - \frac{1}{\sin^2 x} \\
    \end{align*}

    \paragraph{Pochodne funkcji łączonych:}
    \begin{align*}
        (\alpha \cdot f)\prim &= \alpha \cdot f\prim \text{, \quad gdzie} \; \alpha \in \mathbb{R} \text{ jest stałą} \\
        \\
        (f \pm g)\prim &= f\prim \pm g\prim \\
        (f \cdot g)\prim &= f\prim g + f g\prim \\
        (\frac{f}{g})\prim &= \frac{f\prim g - f g\prim}{g^2}
    \end{align*}

    \paragraph[Przydatne pochodne:]
        {Przydatne pochodne\footnote{Przydatne pochodne wywodządze się z pochodnych funkcji podstawowych}:}
    \begin{align*}
        (x)\prim &= 1 \\
        (ax)\prim &= a \\
        (\sqrt{x})\prim &= \frac{1}{2\sqrt{x}} \\
    \end{align*}

    \subsubsection{Tw. Rolle'a}
    Jeżeli
    \begin{enumerate}
        \item funkcja \(f\) jest ciągła w przedziale \([a, b]\),
        \item funckja \(f\) ma pochodną w przedziale \((a, b)\),
        \item \(f(a) = f(b)\)
    \end{enumerate}
    to istnieje \(c \in (a, b)\) takie, że \(f\prim(c) = 0\).

    \subsubsection{Tw. Lagrange'a}
    Jeżeli 
    \begin{enumerate}
        \item funkcja \(f\) jest ciągła w przedziale \([a, b]\),
        \item funckja \(f\) ma pochodną w przedziale \((a, b)\),
    \end{enumerate}
    to istnieje \(c \in (a, b)\) takie, że \(f\prim(c) = \frac{f(b) - f(a)}{b - a}\).

    \subsubsection{Monotoniczność}
    Jeżeli funkcja \(f(x)\) dla każdego \(x \in I\), gdzie \(I\) to dowolny przedział, ma pochodną:
    \begin{enumerate}
        \item \(f\prim(x) = 0\), to funkcja \(f\) jest stała w przedziale \(I\),
        \item \(f\prim(x) > 0\), to funkcja \(f\) jest rosnąca w przedziale \(I\),
        \item \(f\prim(x) < 0\), to funkcja \(f\) jest malejąca w przedziale \(I\),
        \item \(f\prim(x) \geq 0\), to funkcja \(f\) jest niemalejąca w przedziale \(I\),
        \item \(f\prim(x) \leq 0\), to funkcja \(f\) jest nierosnąca w przedziale \(I\),
    \end{enumerate}

    \subsubsection{Ekstrema lokalne}
    \subsubsection*{Warunek konieczny istnienia ekstremum}
    Jeżeli funkcja \(f:(a, b) \rightarrow \mathbb{R}\) jest różniczkowalna i ma w punkcie \(x_0 \in (a,b)\) ekstremum lokalne, to \(f\prim(x_0) = 0\).

    \subsubsection*{Warunek wystarczający istnienia ekstremum}
    Jeżeli funkcja \(f:(a, b) \rightarrow \mathbb{R}\) jest różniczkowalna, \(x_0 \in (a,b)\), \(f\prim(x_0) = 0\) oraz funkcja pochodna \(f\prim\) zmienia znak w otoczeniu punktu \(x_0\), to
    \begin{itemize}
        \item jeżeli \(f\prim\) zmienia znak z \((+)\) na \((-)\), to w punkcie \(x_0\) jest maksimum lokalne,
        \item jeżeli \(f\prim\) zmienia znak z \((-)\) na \((+)\), to w punkcie \(x_0\) jest minimum lokalne,
    \end{itemize}

    \newpage
    \subsubsection{Wypukłość}
    Funkcja jest wypukła w przedziale \(I\), gdy odcinek łączący dowolne dwa punkty wykresu funkcji \(f\) w przedziale \(I\) leży powyżej (funkcja ściśle wypukła) lub na wykresie tej funkcji.

    Analogicznie funcja jest wklęsła gdy odcinek łączący dwa punkty wykresu funkcji \(f\) w przedziale \(I\) leży pod (funkcja ściśle wklęsła) lub na wykresie tej funkcji.

    Jeżeli funkcja \(f(x)\) dla każdego \(x \in I\), gdzie \(I\) to dowolny przedział, ma pochodną drugiego rzędu:
    \begin{enumerate}
        \item \(f\bis(x) > 0\), to funkcja \(f\) jest ściśle wypukła w przedziale \(I\),
        \item \(f\bis(x) < 0\), to funkcja \(f\) jest ściśle wklęsła w przedziale \(I\),
        \item \(f\bis(x) \geq 0\), to funkcja \(f\) jest wypukła w przedziale \(I\),
        \item \(f\bis(x) \leq 0\), to funkcja \(f\) jest wklęsła w przedziale \(I\),
    \end{enumerate}

    \subsubsection{Punkty przegięcia}
    \subsubsection*{Warunek konieczny istnienia punktu przegięcia}
    Jeżeli funkcja \(f\) ma punkt przegięcia w \(x_0\) oraz istnieje pochodna rzędu drugiego funkcji \(f\) w punkcie \(x_0\), to \(f\bis(x_0) = 0\).
    \subsubsection*{Warunek wystarczający istnienia punktu przegięcia}
    Jeżeli funkcja \(f:(a, b) \rightarrow \mathbb{R}\) jest różniczkowalna drugiego rzędu, \(x_0 \in (a,b)\), \(f\bis(x_0) = 0\) oraz funkcja pochodna \(f\bis\) zmienia znak w otoczeniu punktu \(x_0\), to
    \begin{itemize}
        \item jeżeli \(f\bis\) zmienia znak z \((+)\) na \((-)\), albo
        \item jeżeli \(f\bis\) zmienia znak z \((-)\) na \((+)\),
    \end{itemize}
    to istnieje punkt przegięcia.

    \subsection{Pochodne cząstkowe funkcji wielu zmiennych}
    \[f = f(x_1, x_2, \dots, x_n)\]
    Pochodna cząstkowa funkcji \(f\) względem zmiennej \(x_i\)
    \[\frac{\delta f}{\delta x_i} = f\prim_{x_i}\]

    Pochodną cząstkową funkcji \(f\) względem zmiennej \(x_i\) liczymi tak samo jak pochodną funkcji jednej zmiennej, przyjmując, że \(x_i\) to zmienna a wszystkie pozostałe zmienne traktując jak stałe.

    \subsubsection*{Pochodne wyższych rzędów}
    \begin{align*}
        \frac{\delta^2f}{\delta x_j \delta x_i} = f\bis_{x_i x_j} = (f\prim_{x_i})\prim_{x_j} \\
        \frac{\delta^2f}{\delta x^2_i} = f\bis_{x_i x_i} = (f\prim_{x_i})\prim_{x_i}
    \end{align*}

    \subsubsection{Tw. Schwarza}
    Jeżeli pochodne mieszane funkcji \(f(x, y)\) są funkcjami ciągłymi to są sobie równe, czyli
    \[f\bis_{x_i x_j} = f\bis_{x_j x_i}\]

    \subsubsection{Kryterium Sylvestera}
    Kryterium pozwalające badać określoność symetrycznej macierzy.

    Niech
    \begin{displaymath}
        A = 
        \begin{bmatrix}
            a_{1,1} & a_{1,2} & \cdots & a_{1,n} \\
            a_{2,1} & a_{2,2} & \cdots & a_{2,n} \\
            \vdots & \vdots & \ddots & \vdots \\
            a_{n,1} & a_{n,2} & \cdots & a_{n,n} \\
        \end{bmatrix}
    \end{displaymath}
    będzie macierzą symetryczną o współczynnikach rzeczywistych

    Niech ponadto
    \begin{displaymath}
        M_1 = a_{1,1}, \quad 
        M_2 = det
        \begin{bmatrix}
            a_{1,1} & a_{1,2} \\
            a_{2,1} & a_{2,2}
        \end{bmatrix}, \dots \quad
        M_l = det
        \begin{bmatrix}
            a_{1,1} & a_{1,2} & \cdots & a_{1,l} \\
            a_{2,1} & a_{2,2} & \cdots & a_{2,l} \\
            \vdots & \vdots & \ddots & \vdots \\
            a_{l,1} & a_{l,2} & \cdots & a_{n,l} \\
        \end{bmatrix}
    \end{displaymath}

    Wówczas\par
    \(A\) jest dodatnio określona wtedy i tylko wtedy, gdy jej wiodące minory główne są dodatnie, tj.
    \[M_l > 0\;\text{dla}\;l \in \{1,\dots,n\}\]
    \(A\) jest ujemnie określona wtedy i tylko wtedy, gdy 
    \[M_l < 0\;\text{dla}\;l \in \{1, 3, 5, \dots\}\text{,}\; M_l > 0\;\text{dla}\;l \in \{2, 4, 6, \dots\}\]

    \subsubsection{Ekstrema lokalne funkcji dwóch zmiennych}
    \subsubsection*{Warunek konieczny}
    Jeżeli w punkcie \(P_0(x_0, y_0)\) istnieje esktremum to 
    \begin{displaymath}
        \begin{cases}
            f\prim_x(P_0) = 0 \\
            f\prim_y(P_0) = 0
        \end{cases}
    \end{displaymath}
    Punkt w którym spełnione są warunki konieczne jest punktem stacjonarnym.

    \subsubsection*{Warunek wystarczający}
    Jeżeli punkt \(P_0(x_0, y_0)\) jest punktem stacjonarnym oraz niech \(\Delta_1=f\bis_{xx}\) i \(\Delta_2=detf\bis\), gdzie \(f\bis\) to macierz pochodnych cząstkowych drugiego rzędu, to
    \begin{itemize}
        \item jeżeli \(\Delta_1 > 0\) i \(Delta_2 > 0 \implies f\bis\) jest dodatnio określona \(\implies\) minimum lokalne,
        \item jeżeli \(\Delta_1 < 0\) i \(Delta_2 > 0 \implies f\bis\) jest ujemnie określona \(\implies\) maksimum lokalne,
    \end{itemize}

    Uwaga. Jeżeli \(\Delta_2 < 0\) i \(\Delta_1 \neq 0\) to w punkcie stacjonarnym nie ma ekstremum.
\end{document}