\documentclass[../Fizyka.tex]{subfiles}

\begin{document}
    \section{Wprowadzenie}
    \subsection{Wielkości fizyczne}
    Zakres badań fizyki obejmuje wiele przeróżnych obiektów materialnych (ciał, materiałów) i zjawisk (zdarzeń, procesów) występujących w przyrodzie.
    W badaniach podaje się zwykle ich obserwowane cechy, zwane \textbf{wielkościami fizycznymi} — jest to pojęcie podstawowe w fizyce.
    
    Przykłady:
    \begin{itemize}
        \item masa ciała, 
        \item długość, 
        \item czas, 
        \item temperatura, 
        \item objętość,
        \item prędkość ruchu ciała, 
        \item siła, 
        \item natężenie prądu w przewodniku.
    \end{itemize}

    Wyróżniamy badania doświadczalne (eksperyment) i teoretyczne. Wielkości w eksperymencie przedstawiamy przy pomocy \textbf{wartości
    liczbowych} — cechy przedmiotu muszą być \textbf{mierzalne}. Związki pomiędzy wielkościami (prawidłowości, prawa fizyki)
    przedstawiamy przy pomocy \textbf{wzorów matematycznych}. \textbf{Opis matematyczny} oparty o liczby i wzory pozwala na głębsze
    zrozumienie eksperymentu i formułowanie teorii fizycznych. Przewidywania teorii są weryfikowane przez eksperyment.
    Teoria naukowa musi być falsyfikowalna w sensie popperowskim.

    \subsection{Jednostki fizyczne}
    W celu określenia różnic między wartościami danej wielkości musimy wartości te przedstawiać w określonych jednostkach.
    Jednostkowa wielkość - \textbf{jednostka miary} - pozwala otrzymać wartości znormalizowane, które można ze sobą porównywać.
    Jednostka posiada unikalną nazwę i stanowi podstawę miary danej wielkości, np. jednostką długości jest metr, a jej miarą — 10 m.
    \textbf{Prostą wielkość} w eksperymencie porównujemy z wzorcem odnoszącym się do jednej podstawowej jednostki fizycznej.
    Wzorce wielkości podstawowych są dostępne i niezmienne. Miarę \textbf{wielkości złożonej} określa jednostka pochodna na podstawie wzoru matematycznego.
    Wymiar pochodnych wielkości fizycznych przedstawiony jest przy pomocy iloczynu potęg kilku wybranych jednostek podstawowych.
    Obecnie w fizyce przyjmuje się siedem jednostek podstawowych .Wprowadza się również dwie jednostki uzupełniające dot. kąta
    płaskiego (np. radian) i bryłowego (np. steradian).

    \subsection{Pomiar fizyczny}
    W eksperymencie fizycznym źródłem informacji jakościowych jest obserwacja, a ilościowych — pomiar. \textbf{Pomiar fizyczny} prowadzi do wyznaczenia 
    w danym układzie jednostek wartości liczbowej \(\{A\}\) określonej wielkości fizycznej \(A\) przy ustalonej jednostce miary \([A]\).
    Wielkość fizyczna określona jest przez sposób pomiaru lub przez sposób obliczania jej na podstawie innych pomiarów.
    Na przykład droga i czas mogą być zdefiniowane przez określenie metod ich pomiaru, takich jak użycie miarki lub stopera, natomiast
    prędkość średnia może zostać zdefiniowana jako iloraz drogi i czasu.
    Do pomiarów stosujemy \textbf{przyrządy} bezpośrednie lub pośrednie. Przyrządy nie są idealne, mają określoną dokładność.
    Na \textbf{dokładność pomiarową} mają wpływ nieuniknione niepewności pomiarowe oraz usuwalne błędy grube (pomyłki).
    \textbf{Niepewność pomiarową} można ograniczać stosując zarówno doskonalszą aparaturę pomiarową, jak i poprzez optymalizację metody pomiarowej.

    \subsection{Analiza wyników}
    \begin{itemize}
        \item \textbf{Dokładność} \textendash\ dotyczy urządzenia pomiarowego, mówi o jego precyzji,
        \item \textbf{Błąd} \textendash\ to różnica między wartością rzeczywistą i wartością zmierzoną,
        \item \textbf{Niepewność} \textendash\ to statystyczne oszacowanie błędu
        \begin{description}
            \item[Ocena niepewności typu A (przypadkowej)] metody wykorzystujące statystyczną analizę serii pomiarów (obliczanie średnich, regresji itd.)
            \item[Ocena niepewności typu B (systematycznej)] metody wykorzystujące wszystkie informacje o pomiarze oraz źródłach jego niepewności (dokładność przyrządów pomiarowych) 
        \end{description}
    \end{itemize}

    \subsubsection{Analiza wyników metodą typu B}
    \(\Delta x\) \textendash\ Dokładność przyrządu (maksymalny błąd pomiaru),\newline
    \(x\) \textendash\ wartość zmierzona,\newline
    \(\mu\) \textendash\ wartość rzeczywista,\newline
    \(\sigma^2=\frac{(b-a)^2}{12}\) \textendash\ wariancja,\newline
    \(u(x)=\sigma=\frac{\Delta x}{\sqrt{3}}\) \textendash\ niepewność standardowa

    Niepewność standardowa obliczana metodą B jest równa odchyleniu standardowemu.
    
    Funkcja rozkładu gęstości prawdopodobieństwa dla rozkładu jednostajnego:
    \begin{align*}
        \upvarphi(x) &= 
        \begin{cases}
            \frac{1}{2\sigma\sqrt{3}} & x \in [a,b]\\
            0 & x \notin [a,b]
        \end{cases}\\
        a &= \mu - \Delta x\\
        b &= \mu + \Delta x
    \end{align*}
    \begin{figure}[H]
        \centering
        \begin{tikzpicture}
            \begin{axis}[
                xlabel={\(x\)},
                ylabel={\(\upvarphi(x)\)},
                ymin=0, ymax=0.5,
                xmin=0, xmax=7,
                domain=0:7,
                samples=100,
                axis lines=middle,
                every axis plot/.append style={thick},
                ytick={0.3333},
                xtick={1.5,2.34,3.5,4.66,5.5},
                yticklabels={\(\frac{1}{b-a}\)},
                xticklabels={\(a\),\((\mu-\sigma)\),\(\mu\),\((\mu+\sigma)\),\(b\)},
                enlargelimits=false,
                clip=false,
                grid=major,
            ]
            \addplot[domain=1.5:5.5, blue] {1/3};
            \addplot[domain=0:1.5, samples=2, blue] {0};
            \addplot[domain=5.5:7, samples=2, blue] {0};
            \addplot[only marks, mark=*, blue] coordinates {(1.5,1/3) (5.5,1/3)};
            \addplot[only marks, mark=o, blue] coordinates {(1.5,0) (5.5,0)};
            \end{axis}
        \end{tikzpicture}
    \end{figure}

    Wyniki innych pomiarów długości tego samego przedmiotu tym samym przyrządem w \(100\%\) będą się mieścić w przedziale \([a,b]\),
    a w \(58\%\) będą się mieścić w przedziale \((\mu-\sigma,\mu+\sigma)\).

    \textbf{Dowód:} Dla przyrządu z podziałką dokładność przyrządu określa najmniejsza podziałka. Przyjmując, że korzystamy ze zwykłej linijki 
    dokładność przyrządu wynosi \(\Delta x = 0.1cm\). Powierzchnia pod wykresem funkcji określa prawdopodobieństwo, że zmienna losowa przyjmie wartość w określonym przedziale
    co można policzyć za pomocą całki. Dlatego całka oznaczona ww. funkcji w przedziale maksymalnego błędu \([a,b]\) wynosi \(1\):
    \[a = \mu - \Delta x \qquad b = \mu + \Delta x \]
    \[\sigma\sqrt{3} = \frac{\Delta x}{\sqrt{3}} \cdot \sqrt{3} = \Delta x\]
    \[\int \frac{1}{2\sigma\sqrt{3}}dx = \frac{x}{2\sigma\sqrt{3}} + c\]
    \[
        \int\limits_a^b \frac{1}{2\sigma\sqrt{3}}dx =
        \frac{\mu + \Delta x}{2\sigma\sqrt{3}} - \frac{\mu - \Delta}{2\sigma\sqrt{3}} = 
        \frac{\mu + \Delta x - \mu + \Delta x}{2\sigma\sqrt{3}} = 
        \frac{2 \Delta x}{2 \Delta x} = 1 = 100\%
    \]
    Tym samym sposobem można wyliczyć prawdopodobieństwo dla przedziału niepewności standardowej:
    \[
        \int\limits_{\mu-\sigma}^{\mu+\sigma} \frac{1}{2\sigma\sqrt{3}}dx =
        \frac{\mu + \sigma}{2\sigma\sqrt{3}} - \frac{\mu - \sigma}{2\sigma\sqrt{3}} = 
        \frac{\mu + \sigma - \mu + \sigma}{2\sigma\sqrt{3}} = 
        \frac{2 \sigma}{2\sigma\sqrt{3}} = \frac{1}{\sqrt{3}} \approx 0.58 = 58\%
    \]
\end{document}