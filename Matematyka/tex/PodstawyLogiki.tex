\documentclass[../Matematyka.tex]{subfiles}

\begin{document}
    \section{Podstawy logiki matematycznej}
    \paragraph{Zdanie}
    (w logice) jest to wyrażenie w trybie orzekającym, które jest:
    albo \textbf{prawdziwe} - ma wartość logiczną \(\mathbf{1}\),
    albo \textbf{fałszywe} - ma wartość logiczną \(\mathbf{0}\).
    \paragraph{Forma zdaniowa}
    (funkcja zdaniowa, predykat) określona w dziedzinie \(D\) jest 
    to wyrażanie zawierające zmienną (lub zmienne), które staje się zdaniem, gdy 
    w miejsce zmiennej (lub zmiennych) podstawimy nazwę (lub nazwy) 
    dowolnego elementu (lub dowolnych elementów) zbioru \(D\).
    

    \subsection{Funktory zdaniotwórcze}
    \begin{tabular}{lll}
        "Nieprawda, że"                     & - symbol \(\sim\)     & \textbf{Negacja}      \\
        "i"                                 & - symbol \(\land\)    & \textbf{Koniunkcja}   \\
        "lub"                               & - symbol \(\lor\)     & \textbf{Alternatywa}  \\
        "jeżeli \(\dots\), to \(\dots\)"    & - symbol \(\implies\) & \textbf{Implikacja}   \\
        "wtedy i tylko wtedy"               & - symbol \(\iff\)     & \textbf{Równoważność} \\
    \end{tabular}

    \begin{table}[H]
        \centering
        \caption{Wartości logiczne zdań złożonych}
        \begin{tabular}{>{\columncolor[gray]{.8}}c|c|c|>{\columncolor[gray]{.8}}c|c|>{\columncolor[gray]{.8}}c|c}
            \(\sim\!p\) & p & q & \(p \land q\) & \(p \lor q\) & \(p \implies q\) & \(p \iff q\) \\ \hline
            0           & 1 & 1 & 1             & 1            & 1                & 1            \\
            1           & 0 & 1 & 0             & 1            & 1                & 0            \\
                        & 1 & 0 & 0             & 1            & 0                & 0            \\
                        & 0 & 0 & 0             & 0            & 1                & 1            \\
        \end{tabular}
    \end{table}
    
    \subsection{Kwantyfikatory}
    \begin{tabular}{lll}
        "dla każdego \(x \dots\)"            & - symbol \(\underset{x}{\wedge}\) albo \(\underset{x}{\forall}\) & - kwantyfikator duży, ogólny                        \\
        "istnieje \(x\), takie że \(\dots\)" & - symbol \(\underset{x}{\vee}\) albo \(\underset{x}{\exists}\)   & - kwantyfikator mały, szczegółowy, egzystencjonalny \\
    \end{tabular}

    \subsection{Prawa rachunku zdań}

    \textbf{Tautologia} - Zdanie zawsze prawdziwe.

    \subsubsection{Prawa De Morgana dla zdań}
    \subsubsection*{I prawo De Morgana}
    Prawo zaprzeczania koniunkcji: negacja koniunkcji jest równoważna alternatywie negacji
    \[[\sim\!(p \land q)] \iff (\sim\!p \; \lor \sim\!q)\]

    \begin{table}[H]
        \centering
        \caption{Wartości logiczne I prawa De Morgana}
        \begin{tabular}{c|c|>{\columncolor[gray]{.8}}c|c|c|c|>{\columncolor[gray]{.8}}c}
            \(p\) & \(q\) & \(p \land q\) & \(\sim\!(p \land q)\) & \(\sim\!p\) & \(\sim\!q\) & \((\sim\!p) \lor (\sim\!q)\) \\
            \hline
            1 & 1 & 1 & 0 & 0 & 0 & 0 \\
            1 & 0 & 0 & 1 & 0 & 1 & 1 \\
            0 & 1 & 0 & 1 & 1 & 0 & 1 \\
            0 & 0 & 0 & 1 & 1 & 1 & 1 \\
        \end{tabular}
    \end{table}

    \newpage
    \subsubsection*{II prawo De Morgana}
    Prawo zaprzeczenia alternatywy: negacja alternatywy jest równoważna koniunkcji negacji
    \[[\sim\!(p \lor q)] \iff (\sim\!p \; \land \sim\!q)\]

    \begin{table}[H]
        \centering
        \caption{Wartości logiczne II prawa De Morgana}
        \begin{tabular}{c|c|>{\columncolor[gray]{.8}}c|c|c|c|>{\columncolor[gray]{.8}}c}
            \(p\) & \(q\) & \(p \lor q\) & \(\sim\!(p \lor q)\) & \(\sim\!p\) & \(\sim\!q\) & \((\sim\!p) \land (\sim\!q)\) \\
            \hline
            1 & 1 & 1 & 0 & 0 & 0 & 0 \\
            1 & 0 & 1 & 0 & 0 & 1 & 0 \\
            0 & 1 & 1 & 0 & 1 & 0 & 0 \\
            0 & 0 & 0 & 1 & 1 & 1 & 1 \\
        \end{tabular}
    \end{table}

    \subsubsection{Prawo kontrapozycji}
    \[(p \implies q) \iff (\sim\!p \implies \sim\!q)\]

    \begin{table}[H]
        \centering
        \caption{Wartości logiczne prawa kontrapozycji}
        \begin{tabular}{c|c|>{\columncolor[gray]{.8}}c|c|c|>{\columncolor[gray]{.8}}c}
            \(p\) & \(q\) & \(p \implies q\) & \(\sim\!q\) & \(\sim\!p\) & \((\sim\!q) \implies (\sim\!p)\) \\
            \hline
            1 & 1 & 1 & 0 & 0 & 1 \\
            1 & 0 & 0 & 1 & 0 & 0 \\
            0 & 1 & 1 & 0 & 1 & 1 \\
            0 & 0 & 1 & 1 & 1 & 1 \\
        \end{tabular}
    \end{table}

    \subsection{Prawa rachunku kwantyfikatorów}
    Jeżeli \(f(x)\) i \(g(x)\) są formami zdaniowymi o zakresie zmienności \(x \in X\), to:
    \subsubsection*{Prawa De Morgana dla kwantyfikatorów}
    \begin{enumerate}
        \item \(\sim \underset{x}{\forall} f(x) \iff \underset{x}{\exists}\sim f(x)\)
        \item \(\sim \underset{x}{\exists} f(x) \iff \underset{x}{\forall}\sim f(x)\)
    \end{enumerate}
\end{document}