\documentclass[../Matematyka.tex]{subfiles}
\begin{document}
\section{Statystyka}

\subsection{Szeregi statystyczne}
\subsubsection*{Szereg szczegółowy}
\textit{Szeregi szczegółowe} są to uporządkowane ciągi wartości badanej cechy statystycznej.
Wartości takiej cechy mogą być uporządkowane rosnąco:
\[x_1 \leq x_2 \leq \ldots \leq x_n\]
lub malejąco:
\[x_1 \geq x_2 \geq \ldots \geq x_n\]

\subsubsection*{Szereg rozdzielczy}
\textit{Szereg rozdzielczy} stanowi zbiorowość statystyczną podzieloną na części (klasy) według określonej cechy jakościowej lub ilościowej z podaniem liczebości każdej z tych klas. Dla szeregów rozdzielczych cechy ilościowej wyróżniamy:
\begin{itemize}
    \item \textit{szereg rozdzielczy punktowy}
          \begin{table}[H]
              \centering
              \begin{tabular}{c|c|c|c}
                  \(i\)     & \(x_i\)   & \(n_i\)   & \(\sum n_i\) \\ \hline
                  \(1\)     & \(x_1\)   & \(n_1\)   & \(\sum n_1\) \\
                  \(2\)     & \(x_2\)   & \(n_2\)   & \(\sum n_2\) \\
                  \(\dots\) & \(\dots\) & \(\dots\) & \(\dots\)    \\
                  \(k\)     & \(x_k\)   & \(n_k\)   & \(\sum n_k\) \\ \hline\hline
                  \(\sum\)  &           & \(n\)     &              \\
              \end{tabular}
          \end{table}
          gdzie \(x_i\) to wartość cechy, \(n_i\) to liczebność, a \(\sum n_i\) to liczebność skumulowana, tj. \(\sum n_s = n_1 + n_2 + \ldots + n_s\)
    \item \textit{szereg rozdzielczy przedziałowy}
          \begin{table}[H]
              \centering
              \begin{tabular}{c|c|c|c|c}
                  \(i\)     & \(x_{0,i}\) & \(x_{1,i}\) & \(n_i\)   & \(\sum n_i\) \\ \hline
                  \(1\)     & \(x_{0,1}\) & \(x_{1,1}\) & \(n_1\)   & \(\sum n_1\) \\
                  \(2\)     & \(x_{0,2}\) & \(x_{1,2}\) & \(n_2\)   & \(\sum n_2\) \\
                  \(\dots\) & \(\dots\)   & \(\dots\)   & \(\dots\) & \(\dots\)    \\
                  \(k\)     & \(x_{0,k}\) & \(x_{1,k}\) & \(n_k\)   & \(\sum n_k\) \\ \hline\hline
                  \(\sum\)  &             &             & \(n\)     &              \\
              \end{tabular}
          \end{table}
          gdzie \(x_{0,i}\) to dolna granica przedziału, \(x_{1,i}\) to górna granica przedziału, \(n_i\) to liczebność, a \(\sum n_i\) to liczebność skumulowana.
\end{itemize}

\newpage
\subsection{Miary położenia (tendencji centralnej)}
\subsubsection*{Średnia arytmetyczna}
Dla poszczególnych szeregów wyraża się wzorami:
\begin{multicols}{3}
    \begin{center}szcegółowego:\end{center}
    \[\bar{x} = \frac{1}{n} \sum_{i=1}^{n}x_i\]
    \begin{center}rozdzielczego punktowego:\end{center}
    \[\bar{x} = \frac{1}{n} \sum_{i=1}^{k}x_in_i\]
    \begin{center}rozdzielczego przedziałowego:\end{center}
    \[\bar{x} = \frac{1}{n} \sum_{i=1}^{k}\dot{x_i}n_i\]
\end{multicols}
gdzie \(n\) to liczba elementów, \(k\) to liczba klas (przedziałów), \(x_i\) to wartość cechy, \(n_i\) to liczebność, a \(\dot{x_i}\) to środek przedziału, tj. \(\dot{x_i} = \frac{x_{0,i} + x_{1,i}}{2}\).
\subsubsection*{Kwantyle}
{\it Kwantyl \(k_p\) rzędu \(p \in [0,1]\)} to wartość cechy, która dzieli uporządkowany szereg na dwie części: jedna zawiera \(p\cdot100\%\) wartości mniejsze lub równe \(kp\), a druga \((1-p)\cdot100\%\) wartości większe lub równe \(kp\).
\begin{itemize}
    \item {\it dla szeregu szczegółowego:} wyznaczamy pozycje kwantyla \(K = p(n+1)\) następnie wyznaczamy wartośc \(kp\) za pomocą wzoru:
          \[kp =
              \begin{cases}
                  x_K                                                                                          & \text{dla } K \in \N    \\
                  x_{\lfloor K \rfloor} + (K - \lfloor K \rfloor)(x_{\lceil K \rceil} - x_{\lfloor K \rfloor}) & \text{dla } k \notin \N
              \end{cases}\]
    \item {\it dla szeregów rozdzielczych:} wyznaczamy pozycję kwantyla za pomocą wzoru \(K = np\) następnie wyznaczamy pierwszą klasę kwantyla dla której \(\sum n_i \geq K\) i wyliczamy wartość \(kp\) za pomocą wzoru:
          \[k_p=x_{0K} + \frac{K - \sum n_{K-1}}{n_K}\cdot h_K\]
          gdzie \(x_{0K}\) to dolna granica klasy, \(K\) to pozycja kwantyla, \(n_K\) to liczebność klasy kwantyla, \(\sum n_{K-1}\) to liczebność skumulowana klasy poprzedzającej, a \(h_K\) to rozpiętość klasy kwantyla (dla szeregu rozdzielczego punktowego \(h_K = 1\)).
\end{itemize}

{\bf Nazwy szczególnych kwantyli:}
\begin{itemize}
    \item Kwantyl rzędu \(\frac{1}{2}\) to mediana,
    \item Kwantyle rzędu \(\frac{1}{4}, \frac{2}{4}, \frac{3}{4}\) to kwartyle,
    \item Kwantyle rzędu \(\frac{1}{5}, \frac{2}{5}, \frac{3}{5}, \frac{4}{5}\) to kwintyle,
    \item Kwantyle rzędu \(\frac{1}{10}, \frac{2}{10}, \ldots, \frac{9}{10}\) to decyle,
    \item Kwantyle rzędu \(\frac{1}{100}, \frac{2}{100}, \ldots, \frac{99}{100}\) to percentyle,
\end{itemize}
\newpage
\subsubsection*{Mediana}
\begin{itemize}
    \item {\it dla szeregu szczegółowego:}
          \[Me =
              \begin{cases}
                  x_{\frac{n+1}{2}}                             & \text{dla nieparzystego } n \\
                  \frac{x_{\frac{n}{2}} + x_{\frac{n}{2}+1}}{2} & \text{dla parzystego } n
              \end{cases}\]
    \item {\it dla szeregów rozdzielczych} medianę wyznaczamy ze wzoru na kwantyl rzędu \(\frac{1}{2}\).
\end{itemize}
\subsubsection*{Dominanta}
\begin{itemize}
    \item {\it dla szeregu szczegółowego i rozdzielczego punktowego} jest to wartość cechy, która występuje najczęściej.
    \item {\it dla szeregu rozdzielczego przedziałowego} dominantę \(D\) wyznaczamy ze wzoru:
          \[D = x_{0d} + \frac{n_d - n_{d-1}}{(n_d - n_{d-1}) + (n_d - n_{d+1})} \cdot h_d\]
          gdzie \(x_{0d}\) to dolna granica klasy, \(n_d\) to liczebność klasy dominanty, \(n_{d-1}\) to liczebność klasy poprzedzającej, \(n_{d+1}\) to liczebność klasy następującej, a \(h_d\) to rozpiętość klasy dominanty.
\end{itemize}

\subsection{Charakterystyki rozproszenia}
\subsubsection*{Wariancja}
Dla poszczególnych szeregów wyraża się wzorami:
\begin{multicols}{3}
    \begin{center}szcegółowego:\end{center}
    \[s^2=\frac{1}{n}\sum_{i=1}^{n}(x_i - \bar{x})^2\]
    \begin{center}rozdzielczego punktowego:\end{center}
    \[s^2=\frac{1}{n}\sum_{i=1}^{k}(x_i - \bar{x})^2 \cdot n_i\]
    \begin{center}rozdzielczego przedziałowego:\end{center}
    \[s^2=\frac{1}{n}\sum_{i=1}^{k}(\dot{x_i} - \bar{x})^2 \cdot n_i\]
\end{multicols}

\subsubsection*{Odchylenie standardowe}
\[s = \sqrt{s^2}\]

\subsubsection*{Odchylenie ćwiartkowe}
\[Q=\frac{Q_3-Q1}{2}, \quad \text{gdzie } \substack{Q_1\text{ -- kwartyl pierwszy}\\Q_3\text{ -- kwartyl trzeci}}\]

\subsubsection*{Współczynnik zmienności}
\[V = \frac{s}{|\bar{x}|}100\%, \quad \text{gdzie } \bar{x} \neq 0\]

\subsubsection*{Rozstęp}
\[R = x_{max} - x_{min}\]

\subsection{Charakterystyki asymetrii}
{\it Współczynnik asymetrii} to miara skośności rozkładu cechy statystycznej. Wartość współczynnika asymetrii pozwala określić, czy rozkład cechy jest symetryczny, czy też przesunięty w jedną ze stron. Współczynnik asymetrii wyraża się wzorem:
\[A = \frac{\mu_3}{(s)^3}\]
gdzie \(\mu_3\) to trzeci moment centralny, a \(s\) to odchylenie standardowe.

Trzeci moment centralny dla poszczególnych szeregów wyraża się wzorami:
\begin{multicols}{3}
    \begin{center}szcegółowego:\end{center}
    \[s^2=\frac{1}{n}\sum_{i=1}^{n}(x_i - \bar{x})^3\]
    \begin{center}rozdzielczego punktowego:\end{center}
    \[s^2=\frac{1}{n}\sum_{i=1}^{k}(x_i - \bar{x})^3 \cdot n_i\]
    \begin{center}rozdzielczego przedziałowego:\end{center}
    \[s^2=\frac{1}{n}\sum_{i=1}^{k}(\dot{x_i} - \bar{x})^3 \cdot n_i\]
\end{multicols}

Gdy \(A = 0\) to rozkład jest symetryczny, gdy \(A > 0\) to rozkład jest prawostronnie skośny, a gdy \(A < 0\) to rozkład jest lewostronnie skośny.

\subsection{Własności estymatorów}

Def. Ciąg \((\hat{Q}_n)_{n\in\N}\) estymatorów paremetru \(Q\) nazywamy:
\begin{itemize}
    \item zgodnym, jeżeli dla dowolnego \(\varepsilon > 0\) \[\lim_{n\rightarrow\infty} P(|\hat{Q}_n - Q| < \varepsilon) = 1\]
    \item nieobciążonym, jeżeli dla dowolnego \(n \in \N\) \[E(\hat{Q}_n) = Q\]
    \item efektywnym, jeżeli dla każdego \(n \in \N\) estymator \(\hat{Q}_n\) ma najmniejszą wariancję spośród wszystkich estymatorów nieobciążonych
\end{itemize}

Rozkład cechy \(X\) w populacji jest normalny

\begin{center}
    \(\displaystyle{\bar{X}_n = \frac{1}{n}\sum_{i=1}^{n}X_i}\) -- estymator wartości średniej\\
    zgodny, nieobciążony i efektywny

    \(\displaystyle{S^2_n = \frac{1}{n}\sum_{i=1}^{n}(X_i-\bar{X}^2)}\) -- estymator wariancji\\
    zgodny, ale obciążony

    \(\displaystyle{S^2_n = \frac{1}{n-1}\sum_{i=1}^{n}(X_i-\bar{X}^2)}\) -- estymator wariancji\\
    zgodny, nieobciążony i efektywny
\end{center}

\newpage
\subsubsection*{Estymacja przedziałowa}
\begin{multicols}{2}
    Przedziały ufności dla średniej

    \(n\) -- wielkość próby\\
    \(u\) -- poziom ufności
\end{multicols}
Rozkład cechy \(X\) w populacji jest normalny \(\mathcal{N}(m, \sigma)\)
\[\mathcal{U} = [\bar{x}-\Delta, \bar{x}+\Delta] \text{ -- przedział ufności}\]

\begin{enumerate}
    \item \makebox[4.5cm][l]{\(\sigma\) jest znane;} \makebox[3.5cm][l]{\(\Delta = t_u \cdot \frac{\sigma}{\sqrt{n}}\)} gdzie \(\Phi(t_u) = \frac{1+u}{2}\)
    \item \makebox[4.5cm][l]{\(n > 30\);} \makebox[3.5cm][l]{\(\Delta = t_u \cdot \frac{s}{\sqrt{n}}\quad\)} gdzie \(\Phi(t_u) = \frac{1+u}{2}\)
    \item \makebox[4.5cm][l]{\(\sigma\) nie jest znane i \(n \leq 30\);} \makebox[3.5cm][l]{\(\Delta = t_{\alpha, n-1} \cdot \frac{s}{\sqrt{n-1}}\quad\)} gdzie \(\alpha = 1-u\) oraz \(t_{\alpha, n-1}\) wartość krytyczna dla rozkładu t-Studenta
\end{enumerate}
\end{document}