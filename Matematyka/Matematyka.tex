\documentclass[draft]{article}
\usepackage[a4paper, top=1.5in, bottom=1.5in, right=1in, left=1in]{geometry}
\usepackage[bottom]{footmisc}

\usepackage{polski}
\usepackage[polish]{babel}

\usepackage[superscript]{cite}

\usepackage[skip=10pt]{parskip}
\usepackage{pgfplots}
\usepackage{mathtools}
\usepackage{amsfonts}
\usepackage{float}
\usepackage[table]{xcolor} 
\usepackage{enumitem}
\usepackage{multicol}

\usepackage[hidelinks]{hyperref}
\hypersetup{final}

\usepackage{subfiles}

\title{Matematyka}
\date{r.a. 2024/2025}
\author{Michał Nycz}

\pgfplotsset{compat=1.18}

\bibliographystyle{plain}

\begin{document}
\newcommand{\bis}{^{\prime\prime}}
\newcommand{\prim}{^\prime}
\newcommand{\R}{\mathbb{R}}
\newcommand{\N}{\mathbb{N}}

\pagenumbering{gobble}
\maketitle
\tableofcontents
\newpage

\pagenumbering{arabic}

\subfile{tex/PodstawyLogiki.tex}
\newpage

\subfile{tex/TeoriaMnogosci.tex}
\newpage

\subfile{tex/RachunekPrawdopodobienstwa.tex}
\newpage

\subfile{tex/Statystyka.tex}
\newpage

\subfile{tex/Macierze.tex}
\newpage

\subfile{tex/RachunekRozniczkowy.tex}
\newpage

\subfile{tex/RachunekCalkowy.tex}

\nocite{*}
\bibliography{Matematyka}
\end{document}